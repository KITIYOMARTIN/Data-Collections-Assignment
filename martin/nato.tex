\documentclass[12pt,A4paper]{article}
\usepackage{zed-csp,graphicx,color}%from
\begin{document}
\begin{titlepage}

  

\centerline{COLLEGE OF COMPUTING AND INFORMATIC SCIENCES}
\paragraph{•}
\centerline{DEPARTMENT OF COMPUTER SCIENCE\\}
\paragraph{•}

\centerline{COURSEWORK: RESEARCH METHODOLOGY(BIT 2207)\\}
\paragraph{•}

\centerline{LECTURER: MR.ERNEST MWEBAZE}
\paragraph{•}

\centerline{TOPIC\\}IMPACTS OF SOCIAL MEDIA AMONG THE YOUTH ON BEHAVIOR CHANGE: \\ A CASE STUDY OF MAKERERE UNIVERSITY KAMPALA,\\UGANDA  \\
\paragraph{•}
\centerline{COMPILED BY: \
 KITIYO MARTIN}
 \paragraph{•}
\centerline{STUDENT NUMBER : 216020769}
\paragraph{•}
\centerline{REGISTRATION NUMBER:16/U/18802}
\paragraph{•}
\end{titlepage}
\pagenumbering{roman}
\tableofcontents
\newpage
\pagenumbering{arabic}

\section{INTRODUCTION}
This research examines the issues of the relation between social media and its impact on 
behaviour change of the youth. Today, messages can reach audiences and target groups in real 
time and they can generate changes and tendencies. Crowds are becoming more powerful 
through technology, because technology has the ability to unite them.
Social media is the integration of digital media including combinations of electronic texts, 
graphics, moving images and sound into a structured computerized environment that allows 
people to interact with the data for appropriate purposes. The digital environment can include the 
internet, telecoms and interactive digital television.

 

\section{Data}
	
The data collected includes the personal information of the student which includes but not limited to the following:
The student name
The student number,gender, sex,the current loation of the student during the interviews
The college of the student and the program offered at the university.
The  current time and  date of the interviews
The type of the media and social chats the student like to use the most and how the think  about the popularity of the media channel chosed.
The hours the spent  using the media on daily basis

   
\section{METHODS FOR DATA COLLECTION.}
Methods of data collection is the techniques used to extract the required information from a given respondent on the topic of interest. The methods used includes but not limited to the following:
 

\subsection{Interviewing.}
An interview is a conversation where questions are asked and answers are given. The students I talked to were very cooperative and give me all the information I needed.
     
\subsection{Questionnaire.}
a set of printed or written questions with a choice of answers, devised for the purposes of a survey or statistical study. I was able to ask variety of question and the students respoded positively except the few who were not ready to cooperate.

\subsection{Sampling.}
These refers to taking a sample or samples of (something) for analysis. The population of makerere university is too vast but I was able to select some students from atleast every college both ladies and gentlemen for my project.

\subsection{Recording instruments like pnone’s record system and camera}
I was able to use my phone camera and recording system to record the conversation I had with some students even though some were not ready to be recorded for their personal reasons.


\subsection{Photography taking.}
I was able to use my phone to take photos of my respondents.

\section{ANALYSIS.}
The media and social media devices has both positive and negative impacts to the population of the campus students.
The positive impacts includes but not limited to the following :\\
1.	Keeps connections between friends when they’re not always able to see each other when they want to.\\
2.	Social media also keeps you up to date with things that are going on around the world rather than just in your area.\\
3.	It gives youth a place to express themselves in a way that a public place wouldn’t allow us to.\\
4.	It helps to develop social skills, a lot of friendships can stem from a social website.\\
5.	It’s a fun way to interact with your peers, other than seeing them in person
\\
Social media has come along with greater harm than the intended good.\\ Initially it was meant to ease life and communication which goal was met. However, in the long run it has portrayed much negativity due to wrong usage.\\
The negativity of social media includes the following among other instances:\\
It encourages demonstrations as plan talks and decisions are made and communicated in different class groups on whatsapp and facebook platforms for example.\\

Most students pay greater attention to their phones than real time people in their vicinity that is to say they prefer social media chatting than real time chatting and this is more passive then active.\\

Less attention payment in class. Some students are busy on chat in the midst of an ongoing lecture and this conflicts with principle understanding.\\
 It also wastes time as things are not done in their right time.\\

Relationships perish resulting from more attention to phones rather than individuals which creates distance and thus breakups.\\

Some become victims of fraud due to fake business advertisements on the different media platforms.\\
 
Social platforms encourage and teach obscenity as they post pornographic pictures and videos. The youth copy these modes in the disguise of attaining western culture.\\


\section{RESULTS} 

I was able to collect information pertaining my respondents using  my phone and upload them to my server in google engine at https://research-methodology-195212.appspot.com


\section{RECOMMENDATION AND CONCLUSION}

A number of things can be revised to Limit the dangers created  by the social platforms. I would suggest the following measures to be undertaken :

Lectures should limit the use of phones amidst lectures unless for calls urgent enough

Parents can limit the types and forms of gadgets they avail their kids with for instance less expensive but good phones that accomodate a few reasonable apps

Lecturers and administrators should be made part of the class groups, then they could monitor what is ongoing incase of demonstration discussions.

Regards my findings, the rate of media destruction is big enough. It can be corrected though. If everyone would use their phones and gadgets correctly in the right timing of everything., The internet can be an amazing platform for everyone and its beauty can be appreciated.

	
In conclusion, social media can have both a beneficial and negative impact on the youth of my generation. It can help youth prosper in so many different ways. As well as hold them down in various ways. The impact that social media has on us is up for us to decide.
\end{document}